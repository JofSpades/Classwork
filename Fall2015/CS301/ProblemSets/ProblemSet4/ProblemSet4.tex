\documentclass[letterpaper,12pt]{article}
\usepackage{fullpage}
\usepackage{amsmath,amssymb,amsthm}
\usepackage{enumitem}          % gives custom-labeled enumerations
\usepackage{graphics}
\usepackage{mathabx}
\usepackage{tikz}

\usetikzlibrary{arrows,automata,positioning}
% Draw states as black circles filled with gray.
\tikzstyle{every state}=[fill=black!10]
% Set the size of the double circle around accept states.
\tikzstyle{accepting}=[double=black!10,double distance=2pt,outer sep=1pt+\pgflinewidth]
% No text on start arrows.
\tikzstyle{every initial by arrow}=[initial text=]
% Set bend angle, set the label to appear next to the curve, and set
% the arrow style.
\tikzstyle{every edge}=[draw,bend angle=15,auto,>=stealth']

\setlist[enumerate]{label=\textbf{\alph{*}.},listparindent=1em}

\setlength{\topmargin}{0in}
\setlength{\headheight}{0in}
\setlength{\headsep}{0in}
\setlength{\topskip}{0in}

\newcommand{\Z}{\mathbb{Z}}     % Rich man's black board's font.
\newcommand{\F}{\mathbb{F}}
\newcommand{\R}{\mathbb{R}}

% taken from hs's mary.tex, and tracing back to knuth.
\def\dash---{\kern.16667em---\penalty\exhyphenpenalty\hskip.16667em\relax}

\newcommand{\st}{\;\;\text{s.t.}\;\;}
\newcommand{\rev}{^{\scriptscriptstyle\mathcal{R}}}
\newcommand*\comp{^{\mathcal C}}
% Write alphabet symbols in typewriter font as per Sipser.
\newcommand\s[1]{\ensuremath{\mathtt{#1}}}
\newcommand*\langop{\textsc}

\let\oldsetminus\setminus
\renewcommand*\setminus{\mathbin\oldsetminus}

% Set the exercises number and your name.
\title{Problem Set \#4;}
\author{Maxwell Petersen}

% No need to change these 4 lines.
\makeatletter
\newcommand\exercise[1]{\par\vspace{4ex}\normalfont\normalsize\noindent
\textbf{\large Problem #1}\par\nobreak\@afterindentfalse\@afterheading}
\makeatother


\begin{document}
\maketitle

\exercise{1}
Let $M_1 = (Q_1, \Sigma_1, \delta_1, q_1, F_1)$ and $M_2 = (Q_2, \Sigma_2, \delta_2, q_2, F_2)$. Such that $L(M_1) = A$ and  $L(M_2) = B$. Build $M = (Q, \Sigma, \delta, q_0, F)$ such that $L(M) = f(A, B)$.\\
\begin{align*}
Q &=Q_1 \times Q_2 \times \{1,2\}\\
\delta(q, t) &=\delta_1(q, t)\rightarrow q \times t \times 1& \forall q \in Q_1,&\forall t \in \Sigma&\\
\delta(q, t) &=\delta_2(q, t)\rightarrow q \times t \times 2& \forall q \in Q_2,&\forall t \in \Sigma&\\
q_0&= q_1\\
F&= F_1 \times F_2
\end{align*}
This works for any state in $f(A, B)$ because for any state in $Q_2$ a $\Sigma$ symbol can be applied specifically to $q\in Q_1$ to progress the sub DFA $A$ and the inverse is true for the DFA for $B$.

\exercise{2}
Let $M_1 = (Q_1, \Sigma_1, \delta_1, q_1, F_1)$ and $M_2 = (Q_2, \Sigma_2, \delta_2, q_2, F_2)$. Such that $L(M_1) = A$ and  $L(M_2) = B$. Build $M = (Q, \Sigma, \delta, q_0, F)$ such that $L(M) = f(A, B)$.\\
\begin{align*}
Q &=Q_1 \times Q_2 \times \{1,2\}\\
\delta(q, t) &=\delta_1(q, t)\rightarrow q \times t \times 1& \forall q \in Q_1,&\forall t \in \Sigma&\\
\delta(q, t) &=\delta_2(q, t)\rightarrow q \times t \times 2& \forall q \in Q_2,&\forall t \in \Sigma&\\
\delta(q, \epsilon) &= q\\
q_0&: initial state\\
F&= F_1 \times F_2
\end{align*}
This works for any state in $f(A, B)$ because for any state in $Q_2$ a $\Sigma$ symbol can be applied specifically to $q\in Q_1$ to progress the sub DFA $A$ and the inverse is true for the DFA for $B$. Also for any $\epsilon$ symbol the sub DFAs don't progress.

\exercise{3}
Let $M_1 = (Q_1, \Sigma_1, \delta_1, q_1, F_1)$ and $M_2 = (Q_2, \Sigma_2, \delta_2, q_2, F_2)$. Such that $L(M_1) = A$ and  $L(M_2) = B$. Build $M = (Q, \Sigma, \Gamma,\delta, q_0, F)$ such that $L(M) = f(A, B)$.\\
\begin{align*}
Q &=Q_1 \times Q_2 \times \{1,2\}\\
\Gamma &=\{X\}\\
\delta(q_0, \epsilon, \$) &=\{(q_1, \$)\}\\
\delta(q, t, \epsilon) &=q \times t \times 1 \rightarrow X& \forall q \in Q_1, t \in \Sigma^{*}\\
\delta(q, t, X) &=q \times t \times 2 \rightarrow \epsilon& \forall q \in Q_2, t \in \Sigma^{*}\\
q_0&: initial state\\
F&= F_1 \times F_2
\end{align*}
This will keep track of the length of each input string by placing a symbol on the stack and popping it off as the two sub routines run to their final states.

\exercise{4}
To show that $B \setminus A$ is decidable with $A$ being a CFL and $B$ being a regular language we make a decider $D$ in the following form:\\
\begin{align*}
D =& "Given\ input < A, B, w>.\\
&1.Convert\ A\ to\ CNF.\\
&2.Simulate\ A\ on\ w\ for\ 2n-1\ steps,\ n\ being\ the\ amount\ of\ rules\ in\ A.\\
&3.Simulate\ B\ on\ w.\\
&4. Accept\ if\ B\ accepts\ and\ A\ rejects,\ otherwise\ reject."
\end{align*}

\exercise{5}
The language is defined in the following manner:\\
$A = \{ <M> |$ given a representation of a program $M$ accept when $M$ crashes$\}$\\
$A_{TM} \leq_m A$ using the mapping function $f(<M,w>) \mapsto <M>$. Assuming that $A$ is RE then $A_{TM}$ would also be RE which has been proven otherwise creating a contradiction. Meaning $A$ is not RE.\\
\cleardoublepage

\exercise{6}
By definition of $ALL_{TM}$ any input TM is seen to accept $\Sigma^{*}$ meaning that $ALL_{TM}$ is not decidable because for some input TMs will loop while running. Let $A$ be a TM that runs $ALL_{TM}$ to prove that it is RE. $A$ is defined in the following manner:\\
\begin{align*}
A = &"On\ input\ <M>.\\
&1.Run\ EQ_{TM}\ <M, C>\ with\ C\ defined\ as\ so:\\
&\quad C\ ='Accept\ all\ strings.'\\
&2.If\ EQ_{TM}\ accepts,\ accept,\ otherwise\ reject."
\end{align*}
We have proven that $EQ_{TM}$ is RE then by having a TM for $ALL_{TM}$ the compliment of $ALL_{TM}$ is not RE because for the compliment to be RE the TM would have to be a decider which it is not.

\exercise{7}
Assume for contradiction that there a procedure $P$ that on input $<G>$ that will return $<G_{normal}>$ such taht if $G$ and $G'$ are CFG's such that $L(G) = L(G')$then $G_{normal} = G'_{normal}$ and if $L(G)\neq L(G')$ then $G_{normal} \neq G'_{normal}$. Let $A = \{a^nb^m | n,m \geq 0\}$ .Let $A' = A$ applying $P$ to $A$ and $A'$ will yelid us with $G_{normal}$ and $G'_{normal}$ respectively  and to show that the results are not unique:\\
\begin{align*}
G_{normal} = &S \rightarrow aSbT | a\\
&T \rightarrow bT | b\\
\\G'_{normal} = &S \rightarrow aTbS | b\\
&T \rightarrow aT | a
\end{align*}
Both of these CFGs produce $A$ and with no smaller possible CFG we show the contradiction with $G_{normal} \neq G'_{normal}$ with $G = G'$.

\end{document}
